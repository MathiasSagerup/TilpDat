\section{Oppgave 1 (100\%) - Heisprogrammering med stigediagram}\label{sec:3-oppgave1}
Å jobbe med tilpassede datasystemer er ofte en øvelse i å lese datablad, og å jobbe med PLS er på ingen måte noe unntak. En av grunnene til dette er det at tilpassede datasystemer er svært spesialiserte systemer, og i motsetning til generell programmering i en IDE finnes det derfor ofte kun informasjon om disse datasystemene i dokumentasjonen som blir gitt av leverandøren.\\

I denne laboppgaven skal dere derfor bruke de utdelte databladene til å implementere grunnleggende heislogikk, og dette er derfor også en slags introduksjon til den neste laboppgaven i TTK4235, nemlig heislaben. Denne oppgaven inneholder to deler: heisspesifikasjon og datablad. Tanken er at dere skal implementere heisspesifikasjonen i Sysmac og overføre den til vår NX102-9000. Til hjelp så har dere walkthroughen vi gikk gjennom i starten av dette lab-dokumentet, samt den utdelte dokumentasjonen.\\

Denne oppgaven skal gi dere grunnleggende kunnskap om programmering av PLSer med stigediagrammer. Før dere spør læringsassistenter, sjekk ut appendiksen, dokumentasjonen eller kurskompendiet for å se om dere finner svar på spørsmålene deres der.

\subsection{Dokumentasjon}
Dere har fått utdelt to pdfer med dokumentasjon, nemlig "Sysmac Studio Version 1 Operation Manual.pdf" og "Sysmac Kurskompendie.pdf". Førstnevnte er den fullstendige dokumentasjonen til Sysmac, og kan lastes ned fra Omron sin nettside \href{https://assets.omron.eu/downloads/manual/en/v5/w504_sysmac_studio_operation_manual_en.pdf}{HER}, men også i TTK4235 sin GitHub-organisasjon, \href{https://github.com/ITK-TTK4235}{HER}. Den andre er en slags introduksjon til programmet, og finnes også i GitHub-organisasjonen vår. Dersom dere lurer på noe veldig spesifikt ville jeg konsultert dokumentasjonen, mens kurskompendiet kan gi litt praktisk info om hvordan man faktisk bruker programmet. I denne laben forventes det bruk av dokumentasjonen, ettersom det er dette man får tilgang til dersom man skulle jobbet med NX102-9000 ute i industrien.\\

Tidligere i denne laben gikk vi gjennom hvordan man raskt kommer i gang med bruk av Sysmac, og dere trenger nok derfor ikke å gå så nøye gjennom de tre første kapitlene av dokumentasjonen. Kapittel 3 i dokumentasjonen, altså "Section 3 System Design", kan derimot være kjekk å ta en titt på dersom du har trøbbel med selve Sysmac-programmet.\\

Kapittel 4, "Programming", er det dere vil få mest bruk for, og da særlig delkapittel 4-5 "Programming Ladder Diagrams". Her vil dere finne det dere trenger for å programmere stigediagrammer.\\

Ut over dette vil kapittel 5, 6 og 7 kunne være nyttige for henholdsvis oppsett av PLSen, tilkopling til PLSen, samt debugging, dersom dette skulle trengs.\\

\subsection{Heisens funksjonalitet}
Heisfunksjonaliteten skal implementeres gjennom stigeprogrammering, som består i å bygge stigediagram av stigetrinn slik vi gjorde tidligere i denne labteksten. I programmeringsområdet i Sysmac vil man finne byggeblokkene under "Toolbox", som man finner på høyre side av brukergrensesnittet. Her har man en søkefunksjon det kan lønne å benytte seg av. To viktige underkategorier i "Toolbox" er "Timer" og "Ladder Tools", hvor særlig "Rung", "Input" og "Output" er sentrale. Ellers refererer vi til kapittel 4.5 i dokumentasjonen. Eller anbefaler vi å eksperimentere med diverse "Function" og "Function Block" i stigediagrammene deres. Disse dukker altså opp når man høyreklikker på et stigetrinn i programmeringsvinduet. Her har man muligheten til å søke etter kodeord for ting man kan trenge, og det kan være lurt å teste ut kodeord som "logic", "timer", "MOVE", osv. Det er mulig å lese om de ulike byggeklossene direkte i Sysmac, også kan man også trykke på "Help" i menylinja helt øverst, eller se i den utdelte dokumentasjonen.

\subsubsection{Programmeringsoppgaven}
Implementer følgende forenklede algoritme for heisoppførsel:

\begin{itemize}
    \item Heisen starter i en etasje, og kjører deretter i heissjakten.
    \item Heisen stopper og åpner døra i 3 sekunder i hver etasje, og kjører deretter videre i samme retning til en ny etasje nås.
    \item Når heisen har nådd toppetasjen, settes retningen til nedover. 
    \item Når heisen har nådd bunnetasjen settes retningen til oppover. 
    \item Bestillingsknappene skal ikke ha noen virkning i denne forenklede implementeringen.
    \item Når en etasje nås skal det tilhørende etasjelyset lyse helt til en ny etasje nås.
\end{itemize}